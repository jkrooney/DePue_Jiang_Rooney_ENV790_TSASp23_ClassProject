% Options for packages loaded elsewhere
\PassOptionsToPackage{unicode}{hyperref}
\PassOptionsToPackage{hyphens}{url}
%
\documentclass[
]{article}
\usepackage{amsmath,amssymb}
\usepackage{lmodern}
\usepackage{iftex}
\ifPDFTeX
  \usepackage[T1]{fontenc}
  \usepackage[utf8]{inputenc}
  \usepackage{textcomp} % provide euro and other symbols
\else % if luatex or xetex
  \usepackage{unicode-math}
  \defaultfontfeatures{Scale=MatchLowercase}
  \defaultfontfeatures[\rmfamily]{Ligatures=TeX,Scale=1}
\fi
% Use upquote if available, for straight quotes in verbatim environments
\IfFileExists{upquote.sty}{\usepackage{upquote}}{}
\IfFileExists{microtype.sty}{% use microtype if available
  \usepackage[]{microtype}
  \UseMicrotypeSet[protrusion]{basicmath} % disable protrusion for tt fonts
}{}
\makeatletter
\@ifundefined{KOMAClassName}{% if non-KOMA class
  \IfFileExists{parskip.sty}{%
    \usepackage{parskip}
  }{% else
    \setlength{\parindent}{0pt}
    \setlength{\parskip}{6pt plus 2pt minus 1pt}}
}{% if KOMA class
  \KOMAoptions{parskip=half}}
\makeatother
\usepackage{xcolor}
\usepackage[margin=2.54cm]{geometry}
\usepackage{graphicx}
\makeatletter
\def\maxwidth{\ifdim\Gin@nat@width>\linewidth\linewidth\else\Gin@nat@width\fi}
\def\maxheight{\ifdim\Gin@nat@height>\textheight\textheight\else\Gin@nat@height\fi}
\makeatother
% Scale images if necessary, so that they will not overflow the page
% margins by default, and it is still possible to overwrite the defaults
% using explicit options in \includegraphics[width, height, ...]{}
\setkeys{Gin}{width=\maxwidth,height=\maxheight,keepaspectratio}
% Set default figure placement to htbp
\makeatletter
\def\fps@figure{htbp}
\makeatother
\setlength{\emergencystretch}{3em} % prevent overfull lines
\providecommand{\tightlist}{%
  \setlength{\itemsep}{0pt}\setlength{\parskip}{0pt}}
\setcounter{secnumdepth}{-\maxdimen} % remove section numbering
\usepackage{booktabs}
\usepackage{longtable}
\usepackage{array}
\usepackage{multirow}
\usepackage{wrapfig}
\usepackage{float}
\usepackage{colortbl}
\usepackage{pdflscape}
\usepackage{tabu}
\usepackage{threeparttable}
\usepackage{threeparttablex}
\usepackage[normalem]{ulem}
\usepackage{makecell}
\usepackage{xcolor}
\ifLuaTeX
  \usepackage{selnolig}  % disable illegal ligatures
\fi
\IfFileExists{bookmark.sty}{\usepackage{bookmark}}{\usepackage{hyperref}}
\IfFileExists{xurl.sty}{\usepackage{xurl}}{} % add URL line breaks if available
\urlstyle{same} % disable monospaced font for URLs
\hypersetup{
  pdftitle={Visionary Final},
  pdfauthor={Justin DePue, John Rooney, and Tony Jiang},
  hidelinks,
  pdfcreator={LaTeX via pandoc}}

\title{Visionary Final}
\author{Justin DePue, John Rooney, and Tony Jiang}
\date{2023-04-28}

\begin{document}
\maketitle

{
\setcounter{tocdepth}{3}
\tableofcontents
}
\begin{itemize}
\tightlist
\item
  Knitting commands in code chunks:

  \begin{itemize}
  \tightlist
  \item
    \texttt{include\ =\ FALSE} - code is run, but neither code nor
    results appear in knitted file
  \item
    \texttt{echo\ =\ FALSE} - code not included in knitted file, but
    results are
  \item
    \texttt{eval\ =\ FALSE} - code is not run in the knitted file
  \item
    \texttt{message\ =\ FALSE} - messages do not appear in knitted file
  \item
    \texttt{warning\ =\ FALSE} - warnings do not appear\ldots{}
  \item
    \texttt{fig.cap\ =\ "..."} - adds a caption to graphical results
  \end{itemize}
\end{itemize}

\#\#Introduction This study was born out of both curiosity and
necessity. One of our teammates faced higher than expected energy bills
this winter and was suddenly faced with the question of whether to
shiver to save money or spend it on heating bills and be forced to eat
Spaghetti-O's as sustenance. While spring has sprung and summer is on
the horizon, we know that winter is coming and that we should prepare
now in order that history not repeat itself.

As students recently armed with the tools to conduct time series
analysis and forecasting, we realized we could complete our final
project while aiding our teammate with knowledge for the future. Our
study question thus emerged: would it be more cost-effective to heat an
apartment in North Carolina using electricity or a natural-gas powered
heater?

Full of optimism that we could complete a class requirement while doing
some good for the world (for as Marvel taught us, when you help someone,
you help everyone), we set out to find the data that would lead us to
the answer we sought. Our journey led us to that great repository of
energy knowledge, the US Energy Information Administration. There we
found two datasets we felt confident would help us help our teammate:
``North Carolina Price of Natural Gas Delivered to Residential Customers
(Dollars Per Thousand Cubic Feet)'', which contained monthly data from
January 1989 through January 2023, and ``Average Retail Price of
Electricity by State and Sector'' which contained monthly data from
January 2001 through January 2023.

\#\#Data

\begin{table}[!h]

\caption{\label{tab:unnamed-chunk-1}Table 1: Summary Statistics}
\centering
\begin{tabular}[t]{l|r|r|r|r|r|r}
\hline
Variable & Observations & Min & Max & Mean Price & Std. Dev. & Median Price\\
\hline
\cellcolor{gray!6}{Natural Gas (\$/Mcf)} & \cellcolor{gray!6}{410} & \cellcolor{gray!6}{5.54} & \cellcolor{gray!6}{30.43} & \cellcolor{gray!6}{13.78} & \cellcolor{gray!6}{5.64} & \cellcolor{gray!6}{12.54}\\
\hline
Electricity(cents/kWh) & 265 & 7.53 & 13.51 & 10.22 & 1.30 & 10.41\\
\hline
\end{tabular}
\end{table}

\begin{table}[!h]

\caption{\label{tab:unnamed-chunk-2}Sample of Cleaned Natural Gas Data}
\begin{tabular}[t]{l|r}
\hline
year & price\\
\hline
\cellcolor{gray!6}{Jan-1989} & \cellcolor{gray!6}{6.17}\\
\hline
Feb-1989 & 6.30\\
\hline
\cellcolor{gray!6}{Mar-1989} & \cellcolor{gray!6}{6.29}\\
\hline
Apr-1989 & 6.80\\
\hline
\cellcolor{gray!6}{May-1989} & \cellcolor{gray!6}{6.99}\\
\hline
Jun-1989 & 8.02\\
\hline
\cellcolor{gray!6}{Jul-1989} & \cellcolor{gray!6}{8.71}\\
\hline
Aug-1989 & 8.97\\
\hline
\cellcolor{gray!6}{Sep-1989} & \cellcolor{gray!6}{8.68}\\
\hline
Oct-1989 & 7.44\\
\hline
\end{tabular}
\end{table}

\begin{wraptable}{l}{0pt}

\caption{\label{tab:unnamed-chunk-2}Sample of Cleaned Electricity Data}
\centering
\begin{tabular}[t]{l|r}
\hline
my\_date & price\_per\_kWh\\
\hline
\cellcolor{gray!6}{Jan.2001} & \cellcolor{gray!6}{7.53}\\
\hline
Feb.2001 & 7.77\\
\hline
\cellcolor{gray!6}{Mar.2001} & \cellcolor{gray!6}{8.02}\\
\hline
Apr.2001 & 8.00\\
\hline
\cellcolor{gray!6}{May.2001} & \cellcolor{gray!6}{8.22}\\
\hline
Jun.2001 & 8.19\\
\hline
\cellcolor{gray!6}{Jul.2001} & \cellcolor{gray!6}{8.31}\\
\hline
Aug.2001 & 8.35\\
\hline
\cellcolor{gray!6}{Sep.2001} & \cellcolor{gray!6}{8.35}\\
\hline
Oct.2001 & 8.67\\
\hline
\end{tabular}
\end{wraptable}

\#\#Analysis We first created initial time series objects and plotted
them, along with ACF and PACF plots to gain an initial sense of what the
series looked like and what seasonality they may have.

\includegraphics{Final-Project_files/figure-latex/plot initial time series-1.pdf}

\includegraphics{Final-Project_files/figure-latex/plot elec and gas ts-1.pdf}

We then decomposed the time series objects for further analysis.

\includegraphics{Final-Project_files/figure-latex/decompose ts-1.pdf}
\includegraphics{Final-Project_files/figure-latex/decompose ts-2.pdf}

\includegraphics{Final-Project_files/figure-latex/ACF and PACF of electricity and natural gas-1.pdf}
\includegraphics{Final-Project_files/figure-latex/ACF and PACF of electricity and natural gas-2.pdf}
\includegraphics{Final-Project_files/figure-latex/ACF and PACF of electricity and natural gas-3.pdf}
\includegraphics{Final-Project_files/figure-latex/ACF and PACF of electricity and natural gas-4.pdf}

Several models were developed and tested to determine what would best
fit the data we had and what may lead to the best forecast to determine
whether heating via electricity or natural gas would be most
cost-efficient in the upcoming winter. The five that were used were the
Seasonal ARIMA, ARIMA with Fourier terms, Neural Networks, TBATS, and
STL + EST.

Each of these tests used functions from the ``forecast'' library and
required the use of time series data which we created using the ``ts''
function from the ``tseries'' library.

While the Seasonal ARIMA was a logical first model to try, we quickly
felt that the performance was not particularly strong.

\includegraphics{Final-Project_files/figure-latex/Seasonal Arima-1.pdf}
\includegraphics{Final-Project_files/figure-latex/Seasonal Arima-2.pdf}

We decided to explore more advanced models that could handle the ARIMA
with Fourier Terms model, Neural Networks, TBATS, and STL + EST. ARIMA
with Fourier terms is known as a dynamic harmonic regression model with
an ARMA error structure, using the ``fourier'' function from package
``forecast'' to find terms that model seasonal components.

\includegraphics{Final-Project_files/figure-latex/Arima with Fourier terms-1.pdf}
\includegraphics{Final-Project_files/figure-latex/Arima with Fourier terms-2.pdf}

Then we tried STL model
\includegraphics{Final-Project_files/figure-latex/STL -1.pdf}
\includegraphics{Final-Project_files/figure-latex/STL -2.pdf}

Then we tried to use neural network model. We used nnetar() in forecast
package. We figured out p and P arguments in nnetar() have significant
impact on model performance. Therefore, we first tried to identify the
optimal p and P conbination by trying different combinations.

\begin{verbatim}
##            ME      RMSE       MAE        MPE      MAPE      ACF1 Theil's U
## 10 -1.1711526 1.2950538 1.1711526 -10.501171 10.501171 0.2845602 45.324045
## 11 -0.8750556 0.9882386 0.8750556  -7.626764  7.626764 0.1899188  4.788693
## 20 -1.0249802 1.1558415 1.0249802  -9.065548  9.065548 0.2994269 17.292650
## 21 -0.8843782 0.9987094 0.8843782  -7.716575  7.716575 0.2088288  4.797234
## 22 -0.8741890 0.9990543 0.8741890  -7.630598  7.630598 0.2245965  4.603708
## 12 -0.8718298 0.9911932 0.8718298  -7.603434  7.603434 0.1946700  4.533869
## 31 -0.8874391 1.0062794 0.8874391  -7.751384  7.751384 0.2285421  4.625552
\end{verbatim}

\begin{verbatim}
##           ME     RMSE      MAE       MPE     MAPE      ACF1 Theil's U
## 10 -2.290371 3.001971 2.394652 -33.51951 34.91864 0.6283229 18.489672
## 11 -1.797333 2.250825 1.982031 -25.68840 27.91033 0.2724584  2.431628
## 20 -2.470905 3.290321 2.692904 -39.70461 42.48297 0.6806823 12.006741
## 21 -1.911384 2.248048 1.987781 -27.98125 28.78345 0.2551216  2.559254
## 22 -2.017579 2.368672 2.067081 -30.23271 30.76147 0.2751018  2.684193
## 12 -1.904689 2.314263 1.976273 -27.63154 28.38549 0.2766502  2.583952
## 31 -1.915628 2.249301 1.971244 -28.10974 28.70148 0.2603616  2.588153
## 32 -1.943870 2.272423 1.996773 -28.78513 29.34898 0.2658964  2.598376
## 33 -2.074424 2.380260 2.155633 -32.06930 32.91945 0.1891883  2.453748
\end{verbatim}

We found that the combination 11 (p = 1 and P = 1) has the best modeling
performance for electricity series, and the combination 32 (p = 3 and P
= 2) has the best modeling performance for natural gas series. Then we
use this combination to run the neural network model with fourier terms
for elecricity series.

\includegraphics{Final-Project_files/figure-latex/unnamed-chunk-8-1.pdf}
\includegraphics{Final-Project_files/figure-latex/unnamed-chunk-8-2.pdf}

Then we tried TBATS models for electricity and natural gas series
\includegraphics{Final-Project_files/figure-latex/unnamed-chunk-10-1.pdf}
\includegraphics{Final-Project_files/figure-latex/unnamed-chunk-10-2.pdf}

Then we think Ukraine War should have a significant impact on natural
gas price and probably eletricity gas too. Also, temperature should be a
good regressor to include since utility bills normally fluctuate in the
same direction with temperature. Therefore, we created two covariates:
UKRWAR and temperature. \(UKRWAR\) is an indiciator variable with values
of 0 and 1. Months before March 2022 have a value of 0, while months
after and including March 2022 have a value of 1. The reason why we set
the cutoff month at March 2002 despite the war started from last
February is because the impact of the war on monthly natural gas price
in February 2022 should be limited since the war started in late
February. The temperature series is the monthly average temperature of
Raleigh area. This is largest geographic level of historical temperature
data.

After creating all the covariates, we repeated our modeling but with
covariates to improve the accuracy of the models. First we incorporated
covariates to neural network model.

\includegraphics{Final-Project_files/figure-latex/unnamed-chunk-13-1.pdf}
\includegraphics{Final-Project_files/figure-latex/unnamed-chunk-13-2.pdf}

Then we use seasonal arima model with temperature and fourier terms to
model two series. UKRWAR is excluded because R reports no suitable ARIMA
model when UKRWAR is included. Function used: auto.arima(xreg)

\includegraphics{Final-Project_files/figure-latex/unnamed-chunk-15-1.pdf}
\includegraphics{Final-Project_files/figure-latex/unnamed-chunk-15-2.pdf}

\#\#Summary and Conclusions

\hypertarget{use-ets-to-model-not-finished-i-dont-know-how-to-do-this-maybe-we-dont-need-to-include-this}{%
\subsection{Use ETS to model -- not finished -- I don't know how to do
this, maybe we don't need to include
this}\label{use-ets-to-model-not-finished-i-dont-know-how-to-do-this-maybe-we-dont-need-to-include-this}}

\hypertarget{compare-performance-scores-and-generate-tables-for-use}{%
\subsubsection{compare performance scores and generate tables for
use}\label{compare-performance-scores-and-generate-tables-for-use}}

\includegraphics{Final-Project_files/figure-latex/unnamed-chunk-18-1.pdf}
\includegraphics{Final-Project_files/figure-latex/unnamed-chunk-18-2.pdf}

\begin{verbatim}
## The best model for electricity by RMSE is:  STL
\end{verbatim}

\begin{table}[!h]

\caption{\label{tab:unnamed-chunk-18}Forecast Accuracy for NC Residential Electricity Price}
\centering
\begin{tabular}[t]{l|r|r|r|r|r|r|r}
\hline
  & ME & RMSE & MAE & MPE & MAPE & ACF1 & Theil's U\\
\hline
SARIMA & -0.67702 & 0.83764 & 0.70998 & -5.83726 & 6.12612 & 0.25988 & 2.73983\\
\hline
ARIMA with Fourier & -0.69901 & 0.87542 & 0.69901 & -6.04750 & 6.04750 & 0.08222 & 2.82856\\
\hline
\cellcolor{red}{STL} & \cellcolor{red}{-0.58551} & \cellcolor{red}{0.76708} & \cellcolor{red}{0.63447} & \cellcolor{red}{-5.01390} & \cellcolor{red}{5.43949} & \cellcolor{red}{0.25835} & \cellcolor{red}{2.36056}\\
\hline
Neural Network & -1.03461 & 1.19731 & 1.03461 & -9.26279 & 9.26279 & 0.17545 & 2.98240\\
\hline
TBATS & -0.74971 & 0.89063 & 0.75658 & -6.50522 & 6.56626 & 0.23462 & 2.86546\\
\hline
Neural Network with Covariates & -0.95381 & 1.12329 & 0.98901 & -8.49812 & 8.80636 & 0.37915 & 2.63625\\
\hline
SARIMA with Tem and Fourier & -0.67930 & 0.81795 & 0.69585 & -5.83257 & 5.97890 & 0.26220 & 2.92127\\
\hline
\end{tabular}
\end{table}

\begin{verbatim}
## The best model for natural gas by RMSE is:  STL
\end{verbatim}

\begin{table}[!h]

\caption{\label{tab:unnamed-chunk-18}Forecast Accuracy for NC Residential Natural Gas Price}
\centering
\begin{tabular}[t]{l|r|r|r|r|r|r|r}
\hline
  & ME & RMSE & MAE & MPE & MAPE & ACF1 & Theil's U\\
\hline
SARIMA & -1.67902 & 1.97784 & 1.67902 & -23.35174 & 23.35174 & -0.15535 & 1.81675\\
\hline
ARIMA with Fourier & -1.73517 & 2.31199 & 1.83980 & -25.17030 & 26.24979 & 0.57850 & 3.12540\\
\hline
\cellcolor{red}{STL} & \cellcolor{red}{-1.04788} & \cellcolor{red}{1.35599} & \cellcolor{red}{1.11314} & \cellcolor{red}{-14.03428} & \cellcolor{red}{14.80904} & \cellcolor{red}{-0.33348} & \cellcolor{red}{1.28849}\\
\hline
Neural Network & -1.83470 & 2.19935 & 1.91016 & -26.69232 & 27.48516 & 0.44891 & 2.65504\\
\hline
TBATS & -1.72858 & 1.89245 & 1.72858 & -24.53539 & 24.53539 & -0.32231 & 2.08610\\
\hline
Neural Network with Covariates & -1.18195 & 1.63706 & 1.33544 & -14.33293 & 16.32618 & 0.47056 & 2.03497\\
\hline
SARIMA with Tem and Fourier & -1.63275 & 1.79495 & 1.63275 & -21.52013 & 21.52013 & -0.01071 & 1.95239\\
\hline
\end{tabular}
\end{table}

\hypertarget{use-stl-to-model-electricity-and-natural-gas-for-the-next-12-month}{%
\subsubsection{Use STL to model electricity and natural gas for the next
12
month}\label{use-stl-to-model-electricity-and-natural-gas-for-the-next-12-month}}

\includegraphics{Final-Project_files/figure-latex/unnamed-chunk-19-1.pdf}

\end{document}
